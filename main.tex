\documentclass[11pt]{article}

\usepackage{amsmath}
\usepackage{textcomp}
\usepackage[top=0.8in, bottom=0.8in, left=0.8in, right=0.8in]{geometry}
% add other packages here

% put your group number and names in the author field
\title{\bf Exercise 5: An Auctioning Agent for the Pickup and Delivery Problem}
\author{Group \textnumero 3: Robin Clerc, Pierre Vigier}

\begin{document}
\maketitle

\section{The optimization inside the agent}

The aim here is to optimize the repartition of the tasks between the vehicle in a centralized way to be the most cost efficient to be competitive against the opponents. We keep the idea of the previous exercise and improve it to take into account the timeout to take advantage of the whole time we are provided with.
Moreover we construct less neighbors but in a more efficient way ; enabling us to do far more iterations and get a better final solution.

Finally as we detected that our final plan was very sensitive to initial conditions we chose to set a limitation of 5000 iterations but to compute tis several time. It gives us far more stability in the computation of this best cost, it is crucial in order to enable us to set a strong bidding strategy

\section{Bidding strategy}

For each new task we start by computing the marginal cost with respect to the list of task we already have, then we multiply it by a ratio reflecting our strategy.

Our initial ratio is $1+0.01 \frac{timeout_{bid}-timeout_{plan}}{timeout_{bid}}$ Indeed we consider that thanks to more time to compute the final plan we will be able to be even more efficient.

Moreover, we observed that often, once we have a first task, the marginal cost of the next ones is far lower than the first. Our strategy is to be very aggressive for the acquisition of the first task, keep the deficit in memory and compensate it in the next bids. Thanks to it we expect our bid to be still lower than the opponent one as it would be its first task, with a high marginal cost as previously stated.

Furthermore we take into account the last result to update our ratio. If we won the last bid then we are likely to be able to win even if we keep a bigger margin, so we can improve our ratio by 1 percent. On the opposite if we lost the last bid, we should reduce our ratio.

Maybe we should take into account the distribution probability to compute the firsts marginal costs.

% describe in details your bidding strategy. Also, focus on answering the following questions:
% - do you consider the probability distribution of the tasks in defining your strategy? How do you speculate about the future tasks that might be auctions?
% - how do you use the feedback from the previous auctions to derive information about the other competitors?
% - how do you combine all the information from the probability distribution of the tasks, the history and the planner to compute bids?

\section{Results}
% in this section, you describe several results from the experiments with your auctioning agent

\subsection{Experiment 1: Comparisons with dummy agents}
% in this experiment you observe how the results depends on the number of tasks auctioned. You compare with some dummy agents and potentially several versions of your agent (with different internal parameter values). 

\subsubsection{Setting}
% you describe how you perform the experiment, the environment and description of the agents you compare with

\subsubsection{Observations}
% you describe the experimental results and the conclusions you inferred from these results

\vdots

\subsection{Experiment n}
% other experiments you would like to present (for example, varying the internal parameter values)

\subsubsection{Setting}

\subsubsection{Observations}

\end{document}
